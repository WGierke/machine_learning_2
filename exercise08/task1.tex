\subsection*{1a}
\begin{figure}[h]
    \centering
    \includegraphics[width=0.8\linewidth]{hmm.png}
    \caption{Name}
    \label{fig:name}
\end{figure}

\subsection*{1c}

Compute:
\begin{equation}
    P((q_1, q_2 )| (O_1, O_2) = (T, T))
\end{equation}


\begin{align}
    P((q_1, q_2 )| (O_1, O_2) = (T, T)) =
    \frac{P((O_1, O_2) = (T, T) | q_1, q_2) P(q_1, q_2)}
    {P((O_1, O_2) = (T, T))}
\end{align}

Lets first compute $P(q_1, q_2)$:

\begin{center}
\begin{tabular}{c | c | c | c | c}
    $q_1$ & $q_2$ & $P(q_1)$ &  $P(q_2|q_1)$ & $P(q_2| q_1) P(q_1) $\\ \hline
    1       & 1   &   1      &    0.1        &   0.1 \\
    1       & 2   &   1      &    0.9        &   0.9 \\
    2       & 1   &   0      &    -        &   0 \\
    2       & 2   &   0      &    -        &   0 \\
\end{tabular}
\end{center}

Now, we look at $ P((O_1, O_2) = (T, T) | q_1, q_2)$:

\begin{center}
\begin{tabular}{c | c | c | c }
    $q_1$ & $q_2$ & $P(O_1 = T | q_1, q_2)$ & $P(O_2 = T | O_1 = T , q_1, q_2)$ \\ \hline
    1       & 1   & 0.8 & $0.8^2 = 0.64 $\\
    1       & 2   & 0.8 & $0.8 \cdot 0.6 = 0.48$  \\
\end{tabular}
\end{center}

This yields for the numerator $P((O_1, O_2) = (T, T) | q_1, q_2) P(q_1, q_2)$.

\begin{center}
\begin{tabular}{c | c | c }
    $q_1$ & $q_2$ & $P((O_1, O_2) = (T, T) | q_1, q_2) P(q_1, q_2) $ \\ \hline
    1       & 1   & $0.64 \cdot 0.1 = 0.064 $\\
    1       & 2   & $0.48 \cdot 0.9 = 0.432 $  \\
\end{tabular}
\end{center}

If we sum the numerator over $q_1$ and $q_2$, we get the probability of the denominator:
\begin{equation}
    P((O_1, O_2) = (T, T)) = 0.064 + 0.432 = 0.496
\end{equation}

Finally, we can compute the posterior probability $P((q_1, q_2 )| (O_1, O_2) = (T, T))$:
\begin{center}
\begin{tabular}{c | c | c }
    $q_1$ & $q_2$ & $P((q_1, q_2 )| (O_1, O_2) = (T, T))$ \\ \hline
    1       & 1   & $ 0.064 / 0.496 = 4 / 31 $\\
    1       & 2   & $ 0.432 / 0.496 = 2 / 31$ \\
\end{tabular}
\end{center}

